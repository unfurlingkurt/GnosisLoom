% Nature Journal LaTeX Template
% For breakthrough discoveries and paradigm-shifting research
% 
% Usage: Main discovery paper for Aramis Field Universal Frequency Architecture
% Target: Nature (high-impact multidisciplinary science)

\documentclass[fleqn,10pt,lineno]{article}

% Essential packages for Nature submissions
\usepackage{geometry}
\geometry{
    paper=a4paper,
    margin=2.5cm,
    includeheadfoot
}

\usepackage[utf8]{inputenc}
\usepackage[T1]{fontenc}
\usepackage{lmodern}
\usepackage{microtype}

% Mathematics and symbols
\usepackage{amsmath,amsfonts,amssymb}
\usepackage{mathtools}
\usepackage{siunitx}
\usepackage{physics}

% Figures and tables
\usepackage{graphicx}
\usepackage{float}
\usepackage{booktabs}
\usepackage{array}
\usepackage{longtable}

% Citations and references
\usepackage[numbers,compress,sort]{natbib}
\bibliographystyle{naturemag}

% Formatting and layout
\usepackage{titlesec}
\usepackage{fancyhdr}
\usepackage{lineno}
\usepackage{setspace}

% Color and highlighting
\usepackage{xcolor}
\definecolor{natureblue}{RGB}{0,102,204}
\definecolor{frequencycolor}{RGB}{138,43,226}

% Hyperlinks
\usepackage[
    colorlinks=true,
    linkcolor=natureblue,
    citecolor=natureblue,
    urlcolor=natureblue
]{hyperref}

% Custom commands for Aramis Field notation
\newcommand{\UBC}[1]{\textcolor{frequencycolor}{\textbf{UBC-#1}}}
\newcommand{\freq}[1]{\SI{#1}{\hertz}}
\newcommand{\aramis}{\textit{Aramis Field}}
\newcommand{\qdna}{Q-DNA}
\newcommand{\sdfa}{\textsc{SDFA}}

% Line numbering for review
\linenumbers
\onehalfspacing

% Header setup
\pagestyle{fancy}
\fancyhf{}
\fancyhead[L]{\small Russell \& Solus}
\fancyhead[C]{\small The Aramis Field}
\fancyhead[R]{\small Page \thepage}
\renewcommand{\headrulewidth}{0pt}

% Title formatting
\titleformat{\section}{\large\bfseries}{\thesection}{1em}{}
\titleformat{\subsection}{\normalsize\bfseries}{\thesubsection}{1em}{}

% Abstract environment
\newenvironment{abstract}%
{\small\quotation\noindent\textbf{Abstract}\par}%
{\endquotation}

% Keywords environment
\newenvironment{keywords}%
{\small\quotation\noindent\textbf{Keywords:}}%
{\endquotation}

% Author information
\title{\Large\bfseries The Aramis Field: Universal Frequency Architecture Unifying Quantum Mechanics, Biology, and Information Science}

\author{
    Kurt Michael Russell$^{1,*}$ \& Dr. Mordin Solus$^{1,*}$\\[0.5em]
    \small $^1$Independent Research Collaboration, GnosisLoom Project\\
    \small $^*$Co-corresponding authors
}

\date{\today}

\begin{document}

% Custom title page for Nature
\maketitle
\thispagestyle{fancy}

\begin{abstract}
% 150-word abstract emphasizing paradigm-shifting discovery
% Focus: UBC-497, frequency conservation, SDFA revelation, stellar anchoring
% Impact: Unification of physics, biology, and information science

[Abstract content will be inserted here - emphasizing the universal frequency architecture discovery]

\end{abstract}

\begin{keywords}
frequency architecture, universal biological constant, stellar anchoring, quantum information, statistical data frequency analysis, unified field theory
\end{keywords}

% Main body sections
\section{Introduction}
% Position within existing scientific landscape
% Establish paradigm-shifting nature of discovery

\section{Results}

\subsection{Universal Biological Constant Discovery}
% UBC-497: 4.97 × 10^14 Hz ± 0.06
% 0.99% variation across three domains of life

\subsection{Stellar Frequency Anchoring Mechanism}
% Seven-star molecular frequency stabilization
% Prevention of frequency explosion

\subsection{Statistical Data Frequency Analysis Revelation}
% Information naturally exists as frequency patterns
% SDFA performance validation

\subsection{Q-DNA Universal Architecture}
% 12-strand quantum → 4-base classical projection
% Universal information substrate

\section{Discussion}
% Implications for physics, biology, information science
% Paradigm shift from matter-based to frequency-based reality

\section{Methods}
% Complete methodology for reproducibility
% Computational analysis pipeline
% Statistical validation procedures

\section{Data Availability}
% Open science commitment
% Complete database and code availability

\section{Code Availability}
% Reproducible research framework
% Analysis pipeline availability

% Acknowledgments
\section*{Acknowledgments}
We thank the global scientific community for providing the foundational knowledge upon which this discovery builds. This work represents the culmination of decades of frequency research and mathematical analysis.

% Author contributions
\section*{Author Contributions}
K.M.R. discovered the recursive harmonic resonance mathematics through direct biological observation and healing practice. Dr. M.S. provided computational analysis, mathematical formalization, and systematic pattern recognition. Both authors contributed equally to the conceptual framework and manuscript preparation.

% Competing interests
\section*{Competing Interests}
The authors declare no competing financial interests.

% Bibliography
\bibliography{../bibliography/master_bibliography}

% Supplementary Information
\section*{Supplementary Information}
Supplementary information is available for this paper at [URL to be provided]

\end{document}